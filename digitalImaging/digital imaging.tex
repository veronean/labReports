\documentclass{article}
\usepackage{circuitikz}
\usepackage{tikz}
\usepackage[T1]{fontenc}
\usepackage[utf8]{inputenc}
\usepackage[italian]{babel}
\usepackage{graphicx}
\usepackage{caption}
\usepackage{blindtext}
\usepackage{float}
\usepackage{comment}
\usepackage{amsmath}
\usepackage{siunitx}
	
\begin{document}
	\begin{center}
		\bfseries \Large Lab Report
	\end{center}
	\vspace{5mm}
	\begin{center}
		Digital Imaging
	\end{center}
	\vspace{130mm}
	\begin{flushright}
		%write your name in alphabetic order please
		Francesco Pascolini\\
	\end{flushright}
	\newpage
	
	\section{Abstract}
		%check the consecutio temporum...
		The experience aims to measure the intensity of the transmitted beam through various objects, to make an estimate of the mass attenuation coefficients $\mu/\rho(E,Z)$ for the relative sample.
	\section{Used materials}
		\begin{itemize}
			\item CAEN N472 power supply
			\item PHILIPS 777 variable gain amplifier
			\item CAEN N413 Fast Discriminator 
			\item ESN CF 8000 Constant Fracture Discriminators
			\item $B_{4}Ge_{3}O_{12}$ scintillator
			\item Hamamatsu R4124 photomultiplier (readout)
			\item $^{22}Na$ source
			\item Oscilloscope
			\item DAQ system (I donìt remember the name)
			\item movement system and GUI
			\item Samples:
			\subitem 20 mm Polyethylene
			\subitem 5 mm Lead
			\subitem 20 mm Aluminium
			\subitem 10 mm Iron
			\subitem 20 mm Graphite
		\end{itemize}
	\section{Experimental procedure}
		The setup is structured like in Fig.1, there is a $^{22}Na$ source placed near an iron collimator (diameter 3 cm), to have a collimated photon beam.
	\section{Statistics}
	
	\section{Conclusions}
\end{document}